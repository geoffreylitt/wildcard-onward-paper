%%
%% This is file `sample-sigconf.tex',
%% generated with the docstrip utility.
%%
%% The original source files were:
%%
%% samples.dtx  (with options: `sigconf')
%%
%% IMPORTANT NOTICE:
%%
%% For the copyright see the source file.
%%
%% Any modified versions of this file must be renamed
%% with new filenames distinct from sample-sigconf.tex.
%%
%% For distribution of the original source see the terms
%% for copying and modification in the file samples.dtx.
%%
%% This generated file may be distributed as long as the
%% original source files, as listed above, are part of the
%% same distribution. (The sources need not necessarily be
%% in the same archive or directory.)
%%
%% The first command in your LaTeX source must be the \documentclass command.
\documentclass[sigplan,10pt,anonymous,review]{acmart}


\providecommand{\tightlist}{%
  \setlength{\itemsep}{0pt}\setlength{\parskip}{0pt}}

%%
%% \BibTeX command to typeset BibTeX logo in the docs
\AtBeginDocument{%
  \providecommand\BibTeX{{%
    \normalfont B\kern-0.5em{\scshape i\kern-0.25em b}\kern-0.8em\TeX}}}

%% Rights management information.  This information is sent to you
%% when you complete the rights form.  These commands have SAMPLE
%% values in them; it is your responsibility as an author to replace
%% the commands and values with those provided to you when you
%% complete the rights form.

%%
%% Submission ID.
%% Use this when submitting an article to a sponsored event. You'll
%% receive a unique submission ID from the organizers
%% of the event, and this ID should be used as the parameter to this command.
%%\acmSubmissionID{123-A56-BU3}

%%
%% The majority of ACM publications use numbered citations and
%% references.  The command \citestyle{authoryear} switches to the
%% "author year" style.
%%
%% If you are preparing content for an event
%% sponsored by ACM SIGGRAPH, you must use the "author year" style of
%% citations and references.
%% Uncommenting
%% the next command will enable that style.
%%\citestyle{acmauthoryear}

%%
%% end of the preamble, start of the body of the document source.
\begin{document}

%%
%% The "title" command has an optional parameter,
%% allowing the author to define a "short title" to be used in page headers.
\title{Table-Driven Customization of Web Applications}

%%
%% The "author" command and its associated commands are used to define
%% the authors and their affiliations.
%% Of note is the shared affiliation of the first two authors, and the
%% "authornote" and "authornotemark" commands
%% used to denote shared contribution to the research.

\author{Geoffrey Litt}
\affiliation{%
  \institution{Massachusetts Institute of Technology}
  \city{Cambridge, MA}
  \country{USA}
}
\email{glitt@mit.edu}

\author{Daniel Jackson}
\affiliation{%
  \institution{Massachusetts Institute of Technology}
  \city{Cambridge, MA}
  \country{USA}
}
\email{dnj@csail.mit.edu}

%%
%% By default, the full list of authors will be used in the page
%% headers. Often, this list is too long, and will overlap
%% other information printed in the page headers. This command allows
%% the author to define a more concise list
%% of authors' names for this purpose.
% \renewcommand{\shortauthors}{Trovato and Tobin, et al.}

%%
%% The abstract is a short summary of the work to be presented in the
%% article.
\begin{abstract}
  In this paper we propose \emph{table-driven customization,} a new
  paradigm for enabling end users to customize software applications
  without doing traditional programming\emph{.} Users directly
  manipulate a tabular view of the structured data inside the
  application, rather than writing imperative scripts as in most
  customization tools. We extend this simple model with a spreadsheet
  formula language and custom data editing widgets, which provide
  sufficient expressivity to implement many useful customizations.

  We describe Wildcard, a browser extension which demonstrates
  table-driven customization in the context of web applications. Through
  concrete examples, we demonstrate that this paradigm can be used to
  create useful customizations for a variety of real applications. We
  share reflections from our usage of the Wildcard system, including its
  strengths and limitations relative to other customization approaches.
  We further explore how new software architectures might help
  application developers promote this style of end-user customization.
\end{abstract}

%%
%% The code below is generated by the tool at http://dl.acm.org/ccs.cfm.
%% Please copy and paste the code instead of the example below.
%%
%% From HERE
\begin{CCSXML}
<ccs2012>
<concept>
<concept_id>10011007.10011006.10011066.10011069</concept_id>
<concept_desc>Software and its engineering~Integrated and visual development environments</concept_desc>
<concept_significance>500</concept_significance>
</concept>
</ccs2012>
\end{CCSXML}

\ccsdesc[500]{Software and its engineering~Integrated and visual development environments}
% To HERE

%%
%% Keywords. The author(s) should pick words that accurately describe
%% the work being presented. Separate the keywords with commas.
\keywords{end-user programming, software customization, web browser extensions}

%% A "teaser" image appears between the author and affiliation
%% information and the body of the document, and typically spans the
%% page.
%\begin{teaserfigure}
%  \includegraphics[width=\textwidth]{sampleteaser}
%  \caption{Seattle Mariners at Spring Training, 2010.}
%  \Description{Enjoying the baseball game from the third-base
%  seats. Ichiro Suzuki preparing to bat.}
%  \label{fig:teaser}
%\end{teaserfigure}

%%
%% This command processes the author and affiliation and title
%% information and builds the first part of the formatted document.
\maketitle

\hypertarget{introduction}{%
\section{Introduction}\label{introduction}}

There have been many attempts at empowering end users to customize their
software by offering them a simplified programming tool. Some scripting
languages (AppleScript, Chickenfoot) have a friendly syntax that
resembles natural language. Visual programming tools (Mac Automator,
Zapier) eliminate text syntax entirely. Macro recorders (Applescript,
Helena, WebVCR) remove some of the initial programming burden by letting
a user start by concrete demonstrations.

These approaches have many differences, but they all share something in
common: an imperative programming model, with mutable variables,
conditionals and loops. End users ultimately must use these traditional
programming constructs to express their ideas, and the object of
interest is fundamentally a \emph{script}, a sequence of commands.

For decades, we have known about an easier model: \emph{direct
manipulation} \citep{shneiderman1983}, where ``visibility of the object
of interest'' replaces ``complex command language syntax''. Direct
manipulation is widely used in GUIs today, but rarely applied for
software customization. In this work, we ask: what would it look like to
build a software customization interface that deeply relies on direct
manipulation?

We take particular inspiration from spreadsheets and visual database
query interfaces \citep{2020a, bakke2016}. These tools enable users to
construct queries, edit data, and compute derived values through direct
manipulation of the data itself, with intermediate feedback available at
every step. These tools have successfully enabled millions of end users
to view, edit, and compute with data.

We have developed a technique called \emph{table-driven customization}
which applies ideas from visual query interfaces in the context of
software customization. We provide the end user with a table view where
they can directly manipulate the internal state of an application.
Changes in the structured table view result in changes to the original
user interface of the application, enabling the user to customize an
application with immediate feedback.

To demonstrate the viability of this approach, we have developed a
browser extension called Wildcard which implements table-driven
customization in the context of the Web. Wildcard works with existing
web applications by using web scraping techniques to extract relevant
application state into a table and to reify changes back into the
original application UI. In Section~\ref{sec:examples} we present
examples of using Wildcard to implement useful customizations across
many real websites, ranging from sorting lists of data to adding whole
new features to applications.

In Section~\ref{sec:architecture} we explain the architecture of the
Wildcard system, focusing on the key abstraction of the \emph{table
adapter} which manages a single table of data. We explain why some of
these abstractions are general enough to support many customizations
beyond the ones we've built so far in Wildcard---for example, we explore
how applications themselves could be re-architected to expose tabular
data to end users, rather than relying on web scraping techniques.

Table-driven customization embodies several design principles for
building customizable software systems, described in
Section~\ref{sec:design-principles}:

\begin{itemize}
\tightlist
\item
  \emph{Data-oriented programming}: End users should be able to
  customize an application by examining and modifying its data, rather
  than by issuing imperative commands to an API
\item
  \emph{Black-box + semantic}: Typically, ``black-box'' customization
  tools, which don't rely on official extension APIs, resort to offering
  low-level APIs for customization. Instead, we suggest a third-party
  community of semantic wrappers, offering end users a more convenient
  semantic API for customization, even if it wasn't built in by the
  original developer.
\item
  \emph{Gentle slope}: Customization tools should offer a ``gentle
  slope'' from normal use to customization. We explore how our approach
  offers end users opportunities to start customizing software with
  almost no effort, which we hope can make customization a viable part
  of everyday software usage rather than a rare event.
\end{itemize}

Table-driven customization relates to existing work in many areas. Our
goals overlap with many software customization tools, in the browser and
on the desktop. Our methods overlap with direct manipulation interfaces
for working with data, including visual database query systems and
spreadsheets. We explore these connections and more in
Section~\ref{sec:related-work}.

\hypertarget{sec:examples}{%
\section{Examples}\label{sec:examples}}

\emph{This section still todo. Borrow heavily from Convivial paper, but
extend with newer results}

\begin{itemize}
\tightlist
\item
  Just use airbnb + expedia from Convivial paper?
\item
  instacart + hacker news?
\item
  summarize/mention all 12 sites we've used it with
\item
  real usage:

  \begin{itemize}
  \tightlist
  \item
    1 externally contributed adapter
  \item
    flux adapter in actual use by Sergio
  \end{itemize}
\end{itemize}

\hypertarget{sec:architecture}{%
\section{System architecture}\label{sec:architecture}}

The basic data unit in table-driven customization is a Table: an ordered
list of tuples, each of which has an ID and some attributes. The user
always sees a single table in the table editor, but this table is the
result of combining data across one or more underlying tables, much like
viewing the result of a SQL query across multiple database tables. Each
individual table's data is managed by a ``table adapter'' component, and
a query engine coordinates across the various table adapters.

(\emph{Todo: drive this explanation via concrete Instacart example. Add
a figure.}) For example, one table adapter manages scraping data from
the DOM of a site, while another manages extracting data from AJAX
requests. Each table adapter produces a table of data, and the two are
joined together by a shared ID column to produce what looks like a
single table of data extracted from the site. Similarly, when the user
makes annotations on a site, these annotations are handled by a third
table adapter, which manages a data table stored in the browser's local
storage.

\hypertarget{table-adapter-api}{%
\subsection{Table adapter API}\label{table-adapter-api}}

In order to promote uniformity within the system, we've defined a small
abstract API, which all table adapters conform to. (Todo: formalize this
more, make it more specific?)

\emph{Output: data table}:The main purpose of a table adapter is to
return a table representing some data source. The table is
live-updating: a table adapter can produce a new version of its data
table at any point (e.g., updating the data table in response to a
change in a website's DOM). When this happens the new data is pushed to
the editor and the query results are refreshed.

Table editors also accept various incoming commands.

\emph{Input: record edits}:The editor can request to a table adapter to
make an edit to a record. The meaning of making an edit can vary
depending on the adapter: in the local storage adapter, an annotation
can be persisted into local storage; in the DOM scraping adapter, an
edit can represent filling in a form field. An adapter can also mark
values as read-only if it wouldn't be meaningful to edit them; for
example, the DOM scraping adapter typically marks page content as
read-only, except for editable form fields.

\emph{Input: query information}:The query engine also sends each table
adapter information about the entire cross-table query view being
created by the user. This functionality is currently mainly used by the
DOM scraping adapter to make changes to the original web page:

\begin{itemize}
\tightlist
\item
  observes the sort order of the query view to re-order rows in the page
\item
  observes the content of other joined tables to render annotations in
  the page
\item
  observes the currently selected row in the UI, to highlight the
  corresponding row in the page
\end{itemize}

\hypertarget{table-adapters}{%
\subsection{Table adapters}\label{table-adapters}}

Here we describe the three types of table adapters which we have
implemented so far to power the customizations shown above. Then, to
demonstrate the generality of the table adapter paradigm, we describe
three more hypothetical types of table adapters which could extend the
power of the Wildcard system.

\hypertarget{existing-adapters}{%
\subsubsection{Existing Adapters}\label{existing-adapters}}

A \textbf{DOM scraping adapter} extracts data from the DOM of a web
page, and manipulates the DOM to re-order rows, edit form entries, and
inject annotations. A programmer only needs to write a single function
which returns scraped data and pointers to relevant DOM elements; the
Wildcard framework uses this function to implement the rest of the
needed functionality.

(\emph{do we want to call DOM scraping adapter a ``live'' adapter
because its data is also shown elsewhere\ldots?})

An \textbf{AJAX scraping adapter} intercepts AJAX requests made by a web
page, and extracts information from those requests. A programmer writes
a function which specifies how to extract data from an AJAX request, and
the framework handles the details of actually intercepting requests and
calling the programmer-defined function.

The \textbf{local storage adapter} simply stores a table of data in the
browser.

\hypertarget{future-adapters}{%
\subsubsection{Future Adapters}\label{future-adapters}}

we have intentionally designed the table adapter API to be general
enough to encapsulate other types of data and additional functionality
in the future. Here are three concrete examples of such possibilities:

\textbf{Integrated website adapters}: We have taken pains to design a
customization system which is not limited to web scraping as the only
means for integrating with existing sites, and can also accommodate
first party developers adding support for table-driven customization
directly into their own websites. A DOM Scraping table adapter could be
swapped out for such an ``integrated website adapter,'' which implements
the exact same interface as a scraping adapter but by directly accessing
the internal state of the user interface, without needing any scraping
logic.

While we have not yet created a fully operational integrated website
adapter, we think it is possible to create such adapters for existing
frontend web frameworks which result in minimal effort on the part of
the application developers. For example, we have created an early
prototype of a plugin for the Redux state management library, which uses
the Model-View-Update pattern that represents the entire state of a user
interface as a single centralized object. To configure an integrated
website adapter for such an existing application, the user can specify a
function projecting the centralized application state as a table, and
handlers for how data edits to the table should affect the state.

\textbf{Shared storage adapter}: It would be useful to share user
annotations between people and across devices---for example,
collaboratively taking notes with friends on a list of options for
places to stay on Airbnb. The existing Local Storage Adapter could be
extended to share live synchronized data with other users. This could be
achieved through a centralized web server (perhaps an existing service
like Google Sheets capable of storing tabular data), or through P2P
connections which might improve the privacy guarantees available for the
shared annotations.

\textbf{Third party API data adapter}: Currently, the main mechanism for
including data from web APIs in Wildcard is using spreadsheet formulas.
However, a web API could also be wrapped to expose a table API that
would dynamically create tables in response to queries. For example,
when fetching walkability scores for many GPS locations, the query
engine could request a table of walkability scores for various pairs of
latitude and longitude, and the table adapter could dynamically perform
API queries to populate a result table. (\emph{todo: why is this better
than spreadsheet formulas? I don't think it offers any more room for
batching performance optimization because we've already framed formulas
as whole-column, not row specific})

\hypertarget{query-engine}{%
\subsection{Query engine}\label{query-engine}}

The query engine is responsible for coordinating across multiple table
adapters. It joins data across multiple tables and creates a single
result table which is shown to the user through the editor. It also
handles all user interactions and routes appropriate messages to each
table adapter.

First, every query involves a primary DOM scraping table adapter which
associates records in the result with elements in the application's user
interface. At minimum, the primary table adapter needs to return record
IDs and have the ability to manipulate the application's UI. It can also
optionally return data about each record.

Next, additional tables (AJAX data, local storage data) are left joined
by ID. (\emph{todo: discuss IDs here?}) Finally, the result table can be
sorted and filtered by any column.

One way to think of this model is a tiny constrained subset of the SQL
query model. We've found that this simple model has proven sufficient
for meeting the needs of customization in practice, and minimizes the
complexity of supporting more general and arbitrary queries. But because
it fits into the general SQL paradigm, it could theoretically be
extended to support more types of queries.

\begin{itemize}
\tightlist
\item
  formulas: do they go here?
\end{itemize}

\hypertarget{table-editor}{%
\subsection{Table editor}\label{table-editor}}

\begin{itemize}
\tightlist
\item
  Our editor is closer to visual SQL query engines like Sieuferd or
  Airtable than a freeform spreadsheet.
\item
  cell editors
\item
  Note that there could be other entire table editors
\end{itemize}

\hypertarget{sec:design-principles}{%
\section{Design principles}\label{sec:design-principles}}

\hypertarget{data-oriented-programming}{%
\subsection{Data-oriented programming}\label{data-oriented-programming}}

There have been many attempts at simplifying aspects of traditional
programming to enable non-programmer end users to perform customization.
End user scripting languages (AppleScript, VBA, Chickenfoot, CoScripter)
have innovated at the syntax level, aiming to improve ease-of-use.
Visual programming environments (Mac Automator, Zapier) go further by
eliminating text syntax entirely, but maintain the same computational
model. Programming-by-demonstration and macro recording environments
(Applescript, Helena, WebVCR) remove some of the initial programming
burden by starting with concrete demonstrations, but typically require
programming later in the process.

However, all of these approaches share something in common: \textbf{an
imperative programming model}. They all involve a list of commands
executed in sequence, with control flow constructs like conditionals and
loops, and mutable variables. This makes some intuitive sense: many
customizations seem most naturally expressed as a series of commands,
each of which has a side effect on the UI: first enter this value, then
click this button, etc. The result is that end users eventually need to
become familiar with all of these traditional imperative programming
constructs, even if only in diluted form. (todo: bolster the case that
this is a barrier to learning)

Imperative programming is not the only possible model. Spreadsheets have
demonstrated another computational model: data editing via direct
manipulation, combined with pure functional expressions for computation,
with automatic reactive updates provided by the runtime. We call this a
data-oriented model, because the user is focused on editing data values
and producing the correct derived data values, rather than encoding a
sequence of commands. The success of spreadsheets with end user
programmers suggests that finding a way to apply data-oriented
programming model to software customization might enable more people to
modify their software. However, it's not immediately obvious how the
spreadsheet paradigm applies to customization.

A key insight from our work is that, rather than representing
customization as a series of commands issued to an existing application,
we can present a projected view of the application's internal state, and
let the user directly edit that projection as a means of customizing the
application. In some sense, this is similar to the DOM inspector, which
provides a user with an alternate representation of the UI that they can
directly manipulate (e.g., by selecting an element and deleting it from
the DOM tree). But rather than provide a low-level view of the user
interface elements, we instead provide a higher-level view of the
semantic state of the application.

\hypertarget{black-box-semantic}{%
\subsection{Black-box + semantic}\label{black-box-semantic}}

Software customization tools typically fall into one of two categories:
black-box, or semantic.

\textbf{Black-box customization tools} enable customization without
using official extension APIs, enabling a broader range of
customizations on top of more applications. For example, web browser
extensions have demonstrated the utility of customizing websites through
manipulating the DOM, without needing explicit extension APIs to be
built in. However, black-box customization comes with a corresponding
cost: these tools can typically only operate at a low level of
abstraction, e.g.~manipulating user interface elements. This makes it
harder for end users to write scripts, and makes the resulting scripts
more brittle. (todo: support this claim more, provide examples)

\textbf{Semantic customization tools} use explicit extension APIs
provided by the application developer. Examples of this include
accessing a backend web API, or writing a customization in Applescript
that uses an application-specific API. The main benefit is that this
allows the extension author to work with meaningful concepts in the
application domain---``create a new calendar event'' rather than ``click
the button that says new event.''---which makes customizations easier to
build and more robust. However, this style limits the range of
extensions that can be built, to only those which the official plugin
API supports.

(Footnote this paragraph?) This categorization is slightly
oversimplified, and there are existing customization ecosystems that
span both categories. For example, part of the success of browser
extensions stems from the fact that the DOM encourages the use of
standardized semantic UI elements; even if an app developer doesn't
anticipate customization, merely using of semantic and accessible HTML
creates a collateral benefit of easier extension. Another example of a
middle ground is AppleScript, which provides system-wide low-level APIs
for manipulating GUIs as a means of black-box customization, while also
enabling developers to optionally add application-specific APIs to add
support for semantic customization.

With Wildcard, we use a hybrid approach that takes the best of both
worlds. Programmers implement an API wrapper that is internally
implemented as a black-box customization, but externally provides a
semantic interface to the application. End users get an ergonomic and
simplified customization experience, but without the need to depend on a
first-party application developer exposing extension APIs.

One way to view this approach is as introducing a new abstraction
barrier into black-box extension. Typically, a black box customization
script combines two responsibilities: 1) mapping the low-level details
of a user interface to semantic constructs (e.g., using CSS selectors to
find certain page elements), and 2) the actual logic of the specific
customization. (todo: could easily show examples of this from browser
extensions, Chickenfoot, etc) Even though the mapping logic is often
more generic than the specific customization (e.g., finding a given
input element is independent of what text to insert into that element),
the intertwining of these two responsibilities in a single script makes
it very difficult to share the mapping logic across scripts. With
Wildcard we propose a decoupling of these two layers: a
community-maintained mapping layer, shared across many specific
customizations by individual users. This architecture has been
successfully used by projects like Gmail.js, an open source project that
creates a convenient API for browser extensions to interface with the
Gmail web email client.

\hypertarget{gentle-slope}{%
\subsection{Gentle slope}\label{gentle-slope}}

\emph{borrow from convivial paper's discussion of in-place}

\begin{itemize}
\tightlist
\item
  here's where ``automation vs customization'' comes in: automation
  frames it as entering a whole new scripting environment; we frame it
  more as just an alternate UI that you can use.
\item
  this is the heart of the gentle slope
\item
  backend APIs: terrible
\item
  the best would be right inline with the GUI (cite Scotty, instrumental
  work) but this has its own problems
\item
  we settle for a compromise: an alternate structured view. (explicitly
  contrast this clearly)
\item
  the CLI GUI thing
\end{itemize}

\hypertarget{sec:related-work}{%
\section{Related Work}\label{sec:related-work}}

Find a way to organize all this:

\begin{itemize}
\tightlist
\item
  our own workshop paper. since then\ldots{}

  \begin{itemize}
  \tightlist
  \item
    fundamentally rearchitected around table adapters
  \item
    evaluated on many more websites
  \item
    more fully describing how the system works
  \end{itemize}
\item
  database GUIs:

  \begin{itemize}
  \tightlist
  \item
    Sieuferd \citep{bakke2016}
  \item
    Airtable \citep{2020a}
  \item
    other similar tools?
  \item
    Liu \& Jagadish: rules for a spreadsheet algebra for database
    queries \citep{liu2009}
  \item
    mention DIRECT MANIPULATION
  \end{itemize}
\item
  browser extensions
\item
  instrumental interaction, Scotty \citep{eagan2011}, Webstrates
\item
  customization research tools: Chickenfoot, Coscripter

  \begin{itemize}
  \tightlist
  \item
    Wildcard, like Chickenfoot, wants to hide HTML from users. But we
    show a structured data view, whereas Chickenfoot shows nothing
  \end{itemize}
\item
  visual customization tools:

  \begin{itemize}
  \tightlist
  \item
    Mac Automator
  \item
    Zapier
  \item
    IFTTT
  \item
    These are weakly direct manipulation. But usually the thing you
    directly manipulate in these UIs is the \emph{script}: not the
    actual \emph{data} you want to operate on. VERY DIFFERENT.
  \end{itemize}
\item
  desktop customization: Applescript, VBA, COM
\item
  browser dev tools
\item
  database GUIs
\item
  spreadsheets
\item
  ScrAPIr
\end{itemize}

%%
%% The next two lines define the bibliography style to be used, and
%% the bibliography file.
\bibliographystyle{ACM-Reference-Format}
\bibliography{wildcard-onward-bibtex.bib}

\end{document}
\endinput
%%
%% End of file `sample-sigconf.tex'.
